\documentclass{article}

%%%%%%%%%%%% 符号相关 %%%%%%%%%%%%
\usepackage{siunitx}                % 单位宏包
% \usepackage{physics}                % 物理符号
\usepackage{amsmath, amssymb}       % 数学宏包
%% 数学字体
% \usepackage[libertine]{newtxmath}
% utopia  adobe-utopia  garamond  urw-garamond  charter  bitstream-charter
\usepackage[garamond]{mathdesign}
% \usepackage[version=4]{mhchem}      % 化学符号
% \usepackage{bm}                     % 数学加粗
%%%%% 常用核素定义
% \newcommand{\Co}{\ce{^{60}Co}}
% \newcommand{\Cs}{\ce{^{137}Cs}}
% \newcommand{\Am}{\ce{^{241}Am}}
% \newcommand{\Na}{\ce{^{24}Na}}
%%%%% 单位样式
\sisetup{
	separate-uncertainty = true,    % 不确定度样式
	inter-unit-product = \ensuremath{{}\cdot{}}     % 单位之间用·连接
}
%%%%%%%%%%%%%%%%%%%%%%%%%%%%%%%%%%

%%%%%%%%%%%% 样式相关 %%%%%%%%%%%%
\usepackage{float}          % 浮动体样式宏包
\usepackage{booktabs}       % 三线表宏包
% \usepackage{multirow}       % 复杂表格多行宏包
% \usepackage{color}          % 字体颜色宏包
%%%%%%%%%%%%%%%%%%%%%%%%%%%%%%%%%%

%%%%%%%%%%  标题样式宏包 %%%%%%%%%%%
\usepackage{caption}
\captionsetup{labelsep=quad}    % 设置分隔符为空格
%%%%%%%%%%%%%%%%%%%%%%%%%%%%%%%%%%

%%%%%%%%%% 页面大小设置宏包 %%%%%%%%%%%
\usepackage{geometry}
\geometry{a4paper, top=2cm, bottom=3cm}     % 设置纸张大小和页边距
%%%%%%%%%%%%%%%%%%%%%%%%%%%%%%%%%%

%%%%%%%%%%%% 中文设置 %%%%%%%%%%%%
\usepackage[fontset=none]{ctex}
% \usepackage{fontspec}   % 字体设置宏包
\usepackage{xeCJK}
\setmainfont{Times New Roman}
\setmonofont{Courier New}
\setCJKmainfont{Noto Serif CJK SC}[BoldFont=Noto Serif CJK SC Bold]
\setCJKmonofont{Noto Sans Mono CJK SC}[BoldFont=Noto Sans Mono CJK SC Medium]
\setCJKsansfont{Noto Sans CJK SC}[BoldFont=Noto Sans CJK SC Medium]

\newCJKfontfamily\zhsong{Noto Serif CJK SC}[BoldFont=Noto Serif CJK SC Bold]
\newCJKfontfamily\zhhei{Noto Sans CJK SC}[BoldFont=Noto Sans CJK SC Medium]
% \newCJKfontfamily\zhkai[BoldFont={FZHei-B01}]{FZKai-Z03}
% \newCJKfontfamily\zhfs[BoldFont={FZHei-B01}]{FZFangSong-Z02}
%%%%%%%%%%%%%%%%%%%%%%%%%%%%%%%%%%

%%%%%%%%%%% 图片处理相关 %%%%%%%%%%%
\usepackage{graphicx}   % 图形宏包
\usepackage{subfigure}  % 子图宏包
\graphicspath{{img/}}  % 指定图片所在文件夹
%%%%%%%%%%%%%%%%%%%%%%%%%%%%%%%%%%

%%%%%%%%%%% 智能引用宏包 %%%%%%%%%%%% 需要最后引入
\usepackage{cleveref}
\crefname{figure}{图}{图}   % 设置图、表、公式引用前缀
\crefname{table}{表}{表}
\crefname{equation}{式}{式}
%%%%%%%%%%%%%%%%%%%%%%%%%%%%%%%%%%

%%%%%%%%%%%%% 其他 %%%%%%%%%%%%%%
% \usepackage{diagbox}        % 斜角表头
% \usepackage{gbt7714}        % 参考文献国标样式
% \usepackage{autofancyhdr}   % 页眉页脚宏包
% \usepackage{enumitem}       % 列表样式调整
% \usepackage{tikz}           % 绘图包
% \usepackage{subfiles}       % 子文件
% \usepackage{chngcntr}		  % 标签编号
% \numberwithin{equation}{section}      % 章节编号为前缀
% \numberwithin{table}{section}
% \numberwithin{figure}{section}
% \counterwithin{equation}{part}        % 自定义编号
% \renewcommand{\thetable}{\arabic{part}.\arabic{table}}
% \usepackage{snotez}                   % 边注
% \setsidenotes{
%     perpage=true,
%     marginnote=true,
%     text-format+=\zhkai
% }
% \usepackage{titletoc}                   % 目录样式
% \setcounter{tocdepth}{2}
% \titlecontents{section}[2.5em]{\bfseries\zhhei\Large\color{ecolor}}
% {\contentslabel{1.3em}}{}
% {\titlerule*[1pc]{}\contentspage}
% \titlecontents{subsection}[3.8em]{\bfseries\zhhei\color{ecolor}}
% {\contentslabel{2.3em}}{}
% {\titlerule*[0.8pc]{.}\contentspage}
%%%%%%%%%%%%%%%%%%%%%%%%%%%%%%%%

\title{{\bfseries\zhhei }}
\author{}
\date{}
